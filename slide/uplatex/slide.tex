% upLaTeXによるBeamerプレゼンテーションスライド
% 文字コードはUTF-8
\documentclass[dvipdfmx, 12pt]{beamer}  % dvipdfmxを使う
\usepackage{bxdpx-beamer}   % dvipdfmxの為に必要
\usepackage{pxjahyper}      % 日本語で栞を作る為に必要
\renewcommand{\kanjifamilydefault}{\gtdefault}  % 規定をゴシック体にする

\usetheme{metropolis}
\usecolortheme{rose}

\title{はじめてのBeamer}
\subtitle{LaTeXでスライドを作ろう!}
\author{fjnkt98}
\date{\today}
\institute{National Institute of Technologie, Ichinoseki College}

\begin{document}
  \maketitle

  % 目次
  \begin{frame}{目次}
    \tableofcontents
  \end{frame}

  % LaTeXのセクションでスライドの章が作れる
  \section{ブロック環境について}

  \begin{frame}{ブロックのあれこれ}
    \begin{block}{block}
      普通のブロック
    \end{block}

    \begin{alertblock}{alertblock}
      アラートブロック
    \end{alertblock}

    \begin{exampleblock}{exampleblock}
      イグザンプルブロック
    \end{exampleblock}
  \end{frame}

  \begin{frame}
    \frametitle{ブロックではないけど似たようなやつ}
    \begin{definition}
      definition環境.何かの定義を書くやつ.
    \end{definition}

    \begin{example}
      example環境.exampleblockとは別のやつです.
    \end{example}

    \begin{theorem}
      theorem環境.定理を書くやつかな?
    \end{theorem}

    \begin{proof}
      proof環境.証明を書くやつらしい...
    \end{proof}
  \end{frame}

  \section{箇条書きとか}

  \begin{frame}
    \frametitle{箇条書き}
    
    通常の\LaTeX と同じ箇条書き環境が使えます
    \begin{itemize}
      \item ほげ
      \item もげ
      \item うんち!
    \end{itemize}

    \begin{enumerate}
      \item いち
      \item に
      \item さん
    \end{enumerate}
  \end{frame}

  \begin{frame}
    \frametitle{箇条書き}

    入れ子の箇条書きも出来る

    enumerate環境でも同様

    \begin{itemize}
      \item トップレベル
      \begin{itemize}
        \item サブレベル
        \begin{itemize}
          \item サブサブレベル
        \end{itemize}
      \end{itemize}
    \end{itemize}
  \end{frame}

  \begin{frame}
    \frametitle{columns環境}
    columns環境を使うと枠を設けて並べることが出来る

    \begin{columns}
      \begin{column}{.50\linewidth}
        \begin{block}{}
          左
        \end{block}

        \begin{block}
          下に並べることも出来る
        \end{block}
      \end{column}

      \begin{column}{.50\linewidth}
        \begin{block}{}
          右
        \end{block}
      \end{column}
    \end{columns}
  \end{frame}

\end{document}