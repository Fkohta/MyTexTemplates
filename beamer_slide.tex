% upLaTeXによるBeamerプレゼンテーションスライド
% 文字コードはUTF-8
\documentclass[dvipdfmx, 12pt]{beamer}  % dvipdfmxを使う
\usepackage{bxdpx-beamer}   % dvipdfmxの為に必要
\usepackage{pxjahyper}      % 日本語で栞を作る為に必要
\renewcommand{\kanjifamilydefault}{\gtdefault}  % 規定をゴシック体にする

\usetheme{metropolis}
\usecolortheme{rose}

\title{はじめてのBeamer}
\author{fjnkt98}
\date{\today}
\institute{Centre for Modern Beamer Themes}

\begin{document}
  \maketitle

  \section{はじめに}
  \begin{frame}{}
    \tableofcontents
  \end{frame}

  \begin{frame}{スライド}
    これがBeamerスライドだ!
  \end{frame}

  \begin{frame}
    \frametitle{はじめてのスライド}
    \begin{definition}
      1と自分自身しか約数を持たない数を\alert{素数}という.
    \end{definition}
    \begin{example}
     \begin{itemize}
      \item 2 は素数.
      \item 3 も素数.
      \item 4 は素数ではない.
     \end{itemize}
    \end{example}
  \end{frame}

  \begin{frame}
    \frametitle{箇条書き}
    \begin{itemize}
      \item ほげ
      \item もげ
      \item うんち!
    \end{itemize}
    \begin{block}{ブロック}
      これはBlockの中身です.
    \end{block}
  \end{frame}

  \begin{frame}
    \frametitle{ブロック}

  \end{frame}

\end{document}